\documentclass[conference]{IEEEtran}
\IEEEoverridecommandlockouts
% The preceding line is only needed to identify funding in the first footnote. If that is unneeded, please comment it out.
\usepackage{cite}
\usepackage{amsmath,amssymb,amsfonts}
\usepackage{algorithmic}
\usepackage{graphicx}
\usepackage{textcomp}
\usepackage{xcolor}
\def\BibTeX{{\rm B\kern-.05em{\sc i\kern-.025em b}\kern-.08em
    T\kern-.1667em\lower.7ex\hbox{E}\kern-.125emX}}
\begin{document}

\title{A Literature Survey of Game Theoritic Models in Network Security}

\author{\IEEEauthorblockN{Naman Arora}
        \IEEEauthorblockA{Graduate Student, CISE Department}
        University of Florida\\
        Gainesville, Florida \\
        naman.arora@ufl.edu
        }

\maketitle

\begin{abstract}
        Internet has become the backbone of the world economy and lives of people all around the world.
        As internet and its reach grew, so did the need for combating the adversaries that attack such services.
        The article is intended to present a survey of novel approaches in fending off the adversaries in a model network using Game Theoretic approach.
        From taxonomies of the types of game theoretic adversary modelling to stochastic model based combating approach.
\end{abstract}

\begin{IEEEkeywords}
        Game-Theory, Cyber security
\end{IEEEkeywords}

\section{Introduction}
Since the \textit{dot-com bubble}, the internet has been laden with malicious actors motivated by lure of profit and fun.
As the internet becomes the backbone of current economy and the lives of billions of people, serious advancements in methods of detection and advancements of such adversaries are required.
Various groups have published methods and studies leveraging a variety of different techniques on how to predict and stifle such adversarial attacks.
Once such area of active study is modelling the behaviour of adversaries and system administrators using a \textit{Game Theoritic} approach.
% PAPER 1 (Survey)
% PAPER 2 (Stocastic GT in network security)
% Notes
% 1. Explain stocastic model in realtion to the premis of the problem, i.e. use th stocastic model to explain why authors chose it.
% 2. Introduce the formal model of the stocastic game including the \beta and its importance.


%\begin{figure}[htbp]
%\centerline{\includegraphics{fig1.png}}
%\caption{Example of a figure caption.}
%\label{fig}
%\end{figure}

\bibliographystyle{ieeetr}
\bibliography{refs.bib}

\end{document}
